\usepackage{amssymb,amsmath,amsthm}
\usepackage{stmaryrd}
\usepackage{todonotes}
\usepackage{mathpartir}
\usepackage{enumerate}
\usepackage{hyperref}
\usepackage{cmll}

\usepackage[barr]{xy}
\let\mto\to
\let\to\rightarrow

\usepackage{color}
\usepackage{supertabular}
% Theorems
\newtheorem{theorem}{Theorem}
\newtheorem{lemma}[theorem]{Lemma}
\newtheorem{example}[theorem]{Example}
\newtheorem{fact}[theorem]{Fact}
\newtheorem{corollary}[theorem]{Corollary}
\newtheorem{definition}[theorem]{Definition}
\newtheorem{remark}[theorem]{Remark}
\newtheorem{proposition}[theorem]{Proposition}
\newtheorem{notn}[theorem]{Notation}
\newtheorem{observation}[theorem]{Observation}

% Commands that are useful for writing about type theory and programming language design.
%% \newcommand{\case}[4]{\text{case}\ #1\ \text{of}\ #2\text{.}#3\text{,}#2\text{.}#4}
\newcommand{\interp}[1]{\llbracket #1 \rrbracket}
\newcommand{\normto}[0]{\rightsquigarrow^{!}}
\newcommand{\join}[0]{\downarrow}
\newcommand{\redto}[0]{\rightsquigarrow}
\newcommand{\nat}[0]{\mathbb{N}}
\newcommand{\fun}[2]{\lambda #1.#2}
\newcommand{\CRI}[0]{\text{CR-Norm}}
\newcommand{\CRII}[0]{\text{CR-Pres}}
\newcommand{\CRIII}[0]{\text{CR-Prog}}
\newcommand{\subexp}[0]{\sqsubseteq}
%% Must include \usepackage{mathrsfs} for this to work.
%% \newcommand{\powerset}[0]{\mathscr{P}}


% Ott includes.
\input{TNT-ott}

\newcommand{\cat}[1]{\mathcal{#1}}
\newcommand{\uar}[0]{\mathop{\uparrow}}
\newcommand{\dar}[0]{\mathop{\downarrow}}
\newcommand{\st}[1]{\mathsf{st}_{#1}}
\newcommand{\case}[0]{\mathsf{case}}
\newcommand{\cur}[1]{\mathsf{cur}(#1)}
\newcommand{\app}[0]{\mathsf{app}}
\newcommand{\id}[0]{\mathsf{id}}
\newcommand{\ifrName}[1]{#1}
\newcommand{\Hom}[2]{\mathsf{Hom}_{#1}(#2)}
\let\t\relax
\newcommand{\t}[0]{\mathsf{t}}
\newcommand{\m}[1]{\mathsf{m}_{#1}}
\newcommand{\n}[1]{\mathsf{n}_{#1}}
\newcommand{\p}[1]{\mathsf{p}_{#1}}
\newcommand{\q}[1]{\mathsf{q}_{#1}}
\newcommand{\limp}[0]{\multimap}

\title{Adjoint Programming: Termination as an Effect}

\author{Harley Eades III\\
  Augusta University}

\begin{document}

\maketitle

\begin{abstract}
  ... 
\end{abstract}

\section{Introduction}
\label{sec:introduction}

It is a well-known fact about adjunctions that they induce a
monoad/comonad pair on some category.  Recently, adjunctions have been
proposed as an elegant model for resources and effects \cite{??}.
Even more recently, they have been proposed as a model of constructive
modal logic.  However, is it worthwhile to adopt adjunctions directly
into a programming language?  If they are adopted, then how do we
program with them?  In this paper we construct a typed
$\lambda$-calculus extended with an adjoint pair of functors with a
focus on programming.  Instead of working in completely generality we
instead focus on a single concrete, but interesting, example.

The Trellys Project was a large scale project to develop a general
purpose dependently type functional programming language \cite{??}.
The most important language design decision was to have the language
fragmented into two parts: a programmatic fragment and a logical
fragment.  The former is used to write general purpose programs, and
the latter is used to verify the correctness of the programs written
in the programmatic fragment.  This implies that there must be a means
for the logical fragment to take programs of the programmatic fragment
-- which we will call simply programs -- in as input to be able to
write predicates and proofs about those programs, but this must be
done so as to not compromise consistency of the logical fragment.  On
the other side of the coin the programmatic fragment should have the
ability to construct programs using the logical fragment.  This
property is called \textbf{freedom of speech}, because both fragments
have the ability to talk about each other without sacrificing their
integrity.

The running example of this paper is to approach this problem from a
categorical perspective. Suppose we have a cartesian closed category,
$\cat{L}$, and a cartesian closed category with fixpoints, $\cat{P}$.
It is well-known that $\cat{L}$ is a model of the simply-typed
$\lambda$-calculus \cite{?}, and $\cat{P}$ is a model of the
simply-typed $\lambda$-calculus with the $[[Y]]$ combinator \cite{?}.
The category $\cat{L}$ will model our logical fragment, and $\cat{P}$
our programmatic fragment.

Now suppose we have a symmetric monoidal adjunction $\cat{P} : T
\dashv D : \cat{L}$. We can think of this adjunction as a translation
between the two fragments.  It is a well-known fact about adjunctions
that the functor $\dar = DT : \cat{P} \to \cat{P}$ is a comonad, and
the functor $\uar = TD : \cat{L} \to \cat{L}$ is a monad.  The former
treats termination as a comonadic effect allowing the programmatic
fragment to restrict itself to terminating programs, and the latter
treats non-termination as a monadic effect allowing proofs to take in
as input potentially diverging programs.  Thus, we can see freedom of
speech as an adjoint situation.  It can also be shown that we have a
strong monad:
\[
\bfig
\morphism<1000,0>[A \times \uar B`\uar(A \times B);\st{A,B}]
\efig
\]
The previous strength can be defined using the fact that we have a
symmetric monoidal adjunction.

It does make a lot of since treating termination as a comonad.  Every
comonad is equipped with an operation $\varepsilon_X : \dar X \to X$.
We can view this comonad as transportation of terminating programs to
non-terminating programs.  That is, any terminating program is a
perfectly good diverging program.  This is one property the Trellys
project had difficulties accommodating.

One application where this is useful is in programming languages such
as Haskell where programmers by default program in potentially
diverging setting.  However, to gain assurances it is often preferred
to be able to pick and choose which programs one would want to be
terminating, and have it enforced by the type checker.  A termination
comonad would allow the programmer to develop their program under the
comonad to insure termination.

We will construct typed $\lambda$-calculi corresponding to both
$\cat{L}$ and $\cat{P}$ respectively.  Then we will bring these two
calculi together through a syntactic adjunction.  Then we will
consider various programmatic questions.  For example, every program
of type $\dar X$ should terminate, but proofs of type $\uar A$ should
be allowed to diverge.  In addition to these syntactic questions we
discuss novel contributions of these adjoint categorical models.

The first model of this kind is due to Benton \cite{??} who described
a similar situation for linear logic where $\cat{P}$ is a model of
linear logic, and $\cat{L}$ is a model of propositional logic, and
hence, the comonand $! : \cat{P} \to \cat{P}$ treats non-linearity as
a comonadic effect, but from the perspective of $\cat{L}$ linearity is
a monadic effect.  

\section{The Fragments and Their Semantics}
\label{sec:the_fragments_and_their_semantics}

We first must define the two fragments and give their semantics.  The
logic fragment will be kept simple and amounts to the simply-typed
$\lambda$-calculus, but the programmatic fragment will be slightly
more interesting.  Since we are interested in programming it will
contain the natural numbers, with a natural number eliminator in the
form of a case analysis, and the Y-combinator.  Each of the theories
will be presented using the style of Crole \cite{??}.

\subsection{The Logical Fragment}
\label{subsec:the_logical_fragment}

The logical fragment is defined in Figure~\ref{fig:log-type-rel} and
Figure~\ref{fig:log-eq}.
\begin{figure}
  \begin{mathpar}
    \scriptsize
    \TNTdruleLXXAx{} \and
    \TNTdruleLXXTrue{} \and
    \TNTdruleLXXProd{} \and
    \TNTdruleLXXProjOne{} \and
    \TNTdruleLXXProjTwo{} \and
    \TNTdruleLXXFun{} \and
    \TNTdruleLXXApp{}
  \end{mathpar}
  \caption{The Logical Fragment: Typing Relation}
  \label{fig:log-type-rel}
\end{figure}
\begin{figure}
  \begin{mathpar}
    \scriptsize
    \TNTdruleLEqXXAx{} \and
    \TNTdruleLEqXXUnit{} \and
    \TNTdruleLEqXXProjOne{} \and
    \TNTdruleLEqXXProjTwo{} \and
    \TNTdruleLEqXXEtaP{} \and
    \TNTdruleLEqXXProjOneC{} \and
    \TNTdruleLEqXXProjTwoC{} \and
    \TNTdruleLEqXXPairC{} \and
    \TNTdruleLEqXXBeta{} \and
    \TNTdruleLEqXXEta{} \and
    \TNTdruleLEqXXFunC{} \and
    \TNTdruleLEqXXSubst{} \and
    \TNTdruleLEqXXWeak{} \and
    \TNTdruleLEqXXEx{} \and
    \TNTdruleLEqXXRefl{} \and
    \TNTdruleLEqXXSym{} \and
    \TNTdruleLEqXXTrans{}
  \end{mathpar}
  \caption{The Logical Fragment: Equality}
  \label{fig:log-eq}
\end{figure}
The following is a well-known fact \cite{Crole}.
\begin{lemma}
  \label{lemma:CCC-models-L}
  The logical fragment has a sound and complete interpretation into
  cartesian closed categories.
\end{lemma}
% subsection the_logical_fragment (end)

\subsection{The Programmatic Fragment}
\label{subsec:the_programmatic_fragment}
The programmatic fragment is a bit more interesting, and corresponds
to the simply-typed $\lambda$-calculus with natural numbers and the
Y-combinator.  The typing relation is defined in Figure~\ref{fig:prog-typing-rel}, and the terms-in-context inference rules are defined in Figure~\ref{fig:prog-eq}.
\begin{figure}
  \begin{mathpar}
    \scriptsize
    \TNTdrulePXXAx{} \and
    \TNTdrulePXXUnit{} \and
    \TNTdrulePXXZero{} \and
    \TNTdrulePXXSuc{} \and
    \TNTdrulePXXFix{} \and
    \TNTdrulePXXProd{} \and
    \TNTdrulePXXProjOne{} \and
    \TNTdrulePXXProjTwo{} \and
    \TNTdrulePXXFun{} \and
    \TNTdrulePXXApp{} \and
    \TNTdrulePXXCase{}
  \end{mathpar}
  \caption{Programmatic Fragment: Typing Relation}
  \label{fig:prog-typing-rel}
\end{figure}
\begin{figure}
  \begin{mathpar}
    \scriptsize
    \TNTdrulePEqXXAx{} \and
    \TNTdrulePEqXXUnit{} \and
    \TNTdrulePEqXXProjOne{} \and
    \TNTdrulePEqXXProjTwo{} \and
    \TNTdrulePEqXXEtaP{} \and
    \TNTdrulePEqXXProjOneC{} \and
    \TNTdrulePEqXXProjTwoC{} \and
    \TNTdrulePEqXXPairC{} \and
    \TNTdrulePEqXXBeta{} \and
    \TNTdrulePEqXXEta{} \and
    \TNTdrulePEqXXFunC{} \and
    \TNTdrulePEqXXFix{} \and
    \TNTdrulePEqXXFixC{} \and
    \TNTdrulePEqXXSuc{} \and
    \TNTdrulePEqXXCaseB{} \and
    \TNTdrulePEqXXCaseS{} \and
    \TNTdrulePEqXXCaseC{} \and
    \TNTdrulePEqXXSubst{} \and
    \TNTdrulePEqXXWeak{} \and
    \TNTdrulePEqXXEx{} \and
    \TNTdrulePEqXXRefl{} \and
    \TNTdrulePEqXXSym{} \and
    \TNTdrulePEqXXTrans{} 
  \end{mathpar}
  \caption{Programmatic Fragment: Equality}
  \label{fig:prog-eq}
\end{figure}
The natural number eliminator, $\mathsf{case}$, can be mixed with the
Y-combinator to write generally recursive programs on the natural
numbers.  For example, the following defines natural number addition:
\[
  [[fix (\r.\n1.\n2.case n2 {
        0 -> n1,
        suc x -> suc (r n1 x)
      })]]
\]

The question now becomes how do we model $[[fix]]$ and $\mathsf{case}$
categorically?  To model $[[fix]]$ we endow a cartesian closed
category with fixpoints \cite{?}.

\begin{definition}
  \label{def:fixpoints-in-CCC}
  Suppose $\cat{C}$ is a cartesian closed category.  Then we say
  $\cat{C}$ \textbf{has fixpoints} if for any object $X$ of $\cat{C}$,
  there is a morphism $[[fix]]_X : (X \to X) \mto X$ such that the
  following diagram commutes:
  \[
  \bfig
  \qtriangle|amm|<1200,500>[
    X \to X`
    (X \to X) \times X`
    X;
    \langle \id_{X \to X}, [[fix]]_X \rangle`
    [[fix]]_X`
    \mathsf{app}_{X,X}]
  \efig
  \]
\end{definition}
\noindent
The commuting diagram in the above definition models the reduction
rule for $[[fix]]$, and so it is pretty easy to see that this
definition will allow us to model $[[fix]]$ categorically.

Perhaps more interestingly we use a novel approach to modeling
natural numbers with their eliminator.
\begin{definition}
  \label{def:SNNO}
  Suppose $\cat{C}$ is a cartesian closed category.  A
  \textbf{Scott natural number object (SNNO)} is an object $[[Nat]]$ of $\cat{C}$ and
  morphisms $\mathsf{z} : 1 \mto [[Nat]]$ and $[[suc]] : [[Nat]]  \mto [[Nat]]$ of $\cat{C}$,
  such that, for any morphisms $f : [[Y]] \mto X$ and $g : [[Y x Nat]] \mto X$ of $\cat{C}$ there is a unique morphism
  $\case_X : [[Y x Nat]] \mto X$ making the following diagrams commute:
       \[
       \begin{array}{lll}
         \bfig
         \qtriangle|amm|<1000,500>[
           [[Y x Nat]]`
           [[Y x Nat]]`
           X;
           \id_Y \times \mathsf{z}`
           \pi_1;f`
           \case_X]

         \ptriangle(1000,0)|amm|/<-`->`->/<1000,500>[
           [[Y x Nat]]`
           [[Y x Nat]]`
           X;
           \id_Y \times [[suc]]`
           \case_X`
           g ]
         \efig                               
       \end{array}
       \]
       Informally, the two diagrams essentially assert that we can
       define $u$ as follows:
       \begin{center}
         \begin{math}
           \begin{array}{rll}
             u\,y\,0 & = & f\,y\\
             u\,y\,[[(suc x)]] & = &  g\,y\,x
           \end{array}
         \end{math}
       \end{center}
\end{definition}

\noindent This formalization of natural numbers is inspired by the
definition of Scott Numerals \cite{??} where the notion of a case
distinction is built into the encoding. We can think of $[[Y]]$ in the
source object of $\case$ as the type additional inputs that will be
passed to both $f$ and $g$, but we can think of $[[Nat]]$ in the
source object of $\case$ as the type of the scrutiny.  Thus, since in
the base case there is no predecessor, $f$, will not require the
scrutiny, and so it is ignored.  

One major difference between SNNOs and the more traditional natural
number objects is that in the definition of the latter $g$ is defined
by well-founded recursion.  However, SNNOs do not allow this, but in
the presence of fixpoints we are able to regain this feature without
having to bake it into the definition of natural number objects.
However, to allow this we have found that when combining fixpoints and
case analysis to define terminating functions on the natural numbers
it is necessary to uniformly construct the input to both $f$ and $g$
due to the reduction rule of $[[fix]]$.  Thus, we extend the type of
$f$ to $[[Y x Nat]]$, but then ignore the second projection when
reaching the base case.

We now have all the necessary structure to describe the model of the
programmatic fragment.
\begin{definition}
  \label{def:P-model}
  A model of the programmatic fragment, called a
  \textbf{$\cat{P}$-model}, consists of a cartesian closed category,
  $\cat{C}$, with fixpoints and a SNNO.
\end{definition}
\noindent
The following definition gives an interpretation of types and typing
contexts into a $\cat{P}$-model.
\begin{definition}
  \label{def:term-interp}
  Suppose $\cat{P}$ is a $\cat{P}$-model.
  \begin{itemize}
  \item The interpretation of types as objects is as follows:
    \[
    \begin{array}{lll}
      \interp{[[True]]} & = & 1\\
      \interp{[[Nat]]} & = & [[Nat]]\\
      \interp{[[X x Y]]} & = & \interp{[[X]]} \times \interp{[[Y]]}\\
      \interp{[[X -> Y]]} & = & \interp{[[X]]} \to \interp{[[Y]]}\\      
    \end{array}
    \]

  \item The interpretation of typing contexts is as follows:
    \[
    \begin{array}{lll}
      \interp{\cdot} & = & 1\\
      \interp{[[G,x1 : X1]]} & = &
      \interp{[[G]]} \times \interp{[[X1]]}\\
    \end{array}
    \]
  \end{itemize}
  We always assume that the interpretation of typing contexts is
  right-associated.  
\end{definition}

The following result is mostly known, but we give the cases of the
proof that are most interesting.  The cases for SNNOs are novel, but
straightforward given their definition.
\begin{theorem}[Soundness]
  \label{theorem:P-soundness}
  Suppose $\cat{P}$ is a $\cat{P}$-model.
  \begin{itemize}
  \item[i.] If $[[G |-P t : X]]$, then there is a morphism,
    $\interp{[[t]]} : \interp{[[G]]} \mto \interp{[[X]]}$, in
    $\cat{P}$, and
  \item[ii.] If $[[G |-P t = t' : X]]$,
    then $\interp{[[t]]} = \interp{[[t']]}$ in $\cat{P}$.
  \end{itemize}
\end{theorem}
\begin{proof}
  We will prove part two explicitly which depends on part one in such
  away that proving part one and then part two would require a lot of
  repeated constructions.  We proceed by induction on the form of the
  assumed term-in-context derivation, but since the majority of the
  programmatic fragment is well-known to have a sound and complete
  interpretation into a cartesian closed category we only show the
  one of the cases for natural numbers.

  \begin{itemize}
    \begin{report}     
  First, recall that $\beta$-equality can be defined as follows in the
  model -- for any morphisms $f : B \times C \mto D$, $g : A \mto C$,
  and $h : A \to B$:
  \[
  \begin{array}{lllll}
    \langle h;\cur{f},g \rangle; \app_{B,C} & = &  \langle h,g \rangle;(\cur{f} \times \id_C);\app_{B,C}\\
    & = & \langle h, g\rangle;f\\
  \end{array}
  \]

  Using $\beta$-equality we can show what we call $[[fix]]$-equality --
  for any morphisms $f : A \to B$ and $g : B \to C$:
  \[
  \begin{array}{llll}
    \langle \cur{\pi_2;g},f \rangle;\app_{B,C}
    & = & \langle \id_A, f \rangle;\pi_2;g\\
    & = & f;g\\
  \end{array}
  \]
  
  
  \item[] Case.\ \\ 
    \begin{center}
      \begin{math}
        $$\mprset{flushleft}
        \inferrule* [right=\ifrName{Fix}] {
          [[G |-P t : X -> X]]
        }{[[G |-P fix t = t (fix t) : X]]}
      \end{math}
    \end{center}
    By part one we know there is a morphism:
    \[ \interp{[[G]]} \mto^{\interp{t}} \interp{[[X]]} \to \interp{[[X]]} \]
    of $\cat{P}$.  In addition, since any $\cat{P}$-model contains
    fixpoints we know there exists a morphism:
    \[
      (\interp{[[X]]} \to \interp{[[X]]}) \mto^{[[fix]]_{\interp{[[X]]}}} \interp{[[X]]}
    \]
    The interpretation of $[[G |-P fix : (X -> X) -> X]]$ is as follows:
    \[
    \interp{[[G]]} \mto^{\interp{[[G |-P fix : (X -> X) -> X]]} = \cur{\pi_2;[[fix]]_{\interp{[[X]]}}}} (\interp{[[X]]} \to \interp{[[X]]}) \to \interp{[[X]]}
    \]
    We obtain our result by the following equational reasoning:
    \[
    \scriptsize
    \begin{array}{lllll}
      \interp{[[G |-P fix t : X]]}\\
      \,\,\,\,\,\,= \langle \cur{\pi_2;[[fix]]_{\interp{[[X]]}}}, \interp{[[t]]} \rangle;\app_{\interp{[[X]]} \to \interp{[[X]]},\interp{[[X]]}}   & \text{(Definition)}\\
      \,\,\,\,\,\,= \interp{[[t]]};[[fix]]_{\interp{[[X]]}} & \text{($[[fix]]$-equality)}\\
      \,\,\,\,\,\,= \interp{[[t]]};\langle \id_{\interp{[[X]]} \to \interp{[[X]]}}, [[fix]]_{\interp{[[X]]}} \rangle;\app_{\interp{[[X]]},\interp{[[X]]}} & \text{(Fixpoint)}\\
      \,\,\,\,\,\,= \langle \interp{[[t]]}, \interp{[[t]]};[[fix]]_{\interp{[[X]]}} \rangle;\app_{\interp{[[X]]},\interp{[[X]]}} & \text{(Cartesian)}\\
      \,\,\,\,\,\,= \langle \interp{[[t]]}, \langle \cur{\pi_2;[[fix]]_{\interp{[[X]]}}},\interp{[[t]]} \rangle;\app_{\interp{[[X]]} \to \interp{[[X]]},\interp{[[X]]}} \rangle;\app_{\interp{[[X]]},\interp{[[X]]}} & \text{($[[fix]]$-equality)}\\
      \,\,\,\,\,\,= \langle \interp{[[t]]}, \langle \interp{[[G |-P fix : (X -> X) -> X]]},\interp{[[t]]} \rangle;\app_{\interp{[[X]]} \to \interp{[[X]]},\interp{[[X]]}} \rangle;\app_{\interp{[[X]]},\interp{[[X]]}} & \text{(Definition)}\\
      \,\,\,\,\,\,= \langle \interp{[[t]]}, \interp{[[G |-P fix t : X]]} \rangle;\app_{\interp{[[X]]},\interp{[[X]]}} & \text{(Definition)}\\
      \,\,\,\,\,\,= \interp{[[G |-P t (fix t) : X]]} & \text{(Definition)}\\      
    \end{array}
     \]

  \item[] Case.\ \\ 
    \begin{center}
      \begin{math}
        $$\mprset{flushleft}
        \inferrule* [right=\ifrName{CaseB}] {
          [[G |-P t1 : X && G, x : Nat |-P t2 : X]]
        }{[[G |-P case 0 {0 -> t1, suc x -> t2} = t1 : X]]}
      \end{math}
    \end{center}
    We have the following morphisms by part one:
    \[
    \begin{array}{c}
      \interp{[[G]]} \mto^{\interp{[[t1]]}} \interp{[[X]]}\\
      \\
      \interp{[[G]]} \times [[Nat]] \mto^{\interp{[[t2]]}} \interp{[[X]]}
    \end{array}
    \]
    Then by the definition of SNNO we know there exists an unique
    morphism
    \[ [[Y x Nat]] \mto^{\mathsf{case}_{\interp{[[X]]}}} \interp{[[X]]}\]
    such that the following is one equation that holds:
    \[
    \begin{array}{lll}
      (\id_{\interp{[[G]]}} \times \mathsf{z});\mathsf{case}_{\interp{[[X]]}} = \pi_1;\interp{[[t1]]}
    \end{array}
    \]
    
    We also have the following interpretation for any $\interp{[[G |-P t : Nat]]}$:
    \[
    \interp{[[case t {0 -> t1, suc x -> t2}]]} =
    \langle \id_{\interp{[[G]]}}, \interp{[[t]]} \rangle;\mathsf{case}_{\interp{[[X]]}} : \interp{[[G]]} \to \interp{[[X]]}
    \]
    Thus, we obtain our desired equality because $\interp{[[G |-P 0 : Nat]]} = \t_{\interp{[[G]]}};\mathsf{z}$:
    \[
    \begin{array}{lll}
      \langle \id_{\interp{[[G]]}}, \t_{\interp{[[G]]}};\mathsf{z} \rangle;\case_{\interp{[[X]]}}
      & = & \langle \id_{\interp{[[G]]}}, \t_{\interp{[[G]]}} \rangle;(\id_{\interp{[[G]]}} \times \mathsf{z});\case_{\interp{[[X]]}}\\
      & = & \langle \id_{\interp{[[G]]}}, \t_{\interp{[[G]]}} \rangle;\pi_1;\interp{[[t1]]}\\
      & = & \interp{[[t1]]}.\\
    \end{array}
    \]
\end{report}
  \item[] Case.\ \\ 
    \begin{center}
      \begin{math}
        $$\mprset{flushleft}
        \inferrule* [right=\ifrName{CaseS}] {
          [[G |-P n : Nat]]
          \\\\
          [[G |-P t1 : X && G, x : Nat |-P t2 : X]]
        }{[[G |-P case (suc n) {0 -> t1, suc x -> t2} = [n/x]t2 : X]]}
      \end{math}
    \end{center}
    We have the following morphisms by part one:
    \[
    \begin{array}{c}
      \interp{[[G]]} \mto^{\interp{[[n]]}} [[Nat]]\\
      \\
      \interp{[[G]]} \mto^{\interp{[[t1]]}} \interp{[[X]]}\\
      \\
      \interp{[[G]]} \times [[Nat]] \mto^{\interp{[[t2]]}} \interp{[[X]]}
    \end{array}
    \]
    Then by the definition of SNNO we know there exists an unique
    morphism
    \[ [[Y x Nat]] \mto^{\mathsf{case}_{\interp{[[X]]}}} \interp{[[X]]}\]
    such that the following is one equation that holds:
    \[
    \begin{array}{lll}
      (\id_{\interp{[[G]]}} \times \mathsf{suc});\mathsf{case}_{\interp{[[X]]}} = \interp{[[t2]]}
    \end{array}
    \]
    
    We also have the following interpretation for any $\interp{[[G |-P t : Nat]]}$:
    \[
    \interp{[[case t {0 -> t1, suc x -> t2}]]} =
    \langle \id_{\interp{[[G]]}}, \interp{[[t]]} \rangle;\mathsf{case}_{\interp{[[X]]}} : \interp{[[G]]} \to \interp{[[X]]}
    \]
    We obtain our result because $\interp{[[G |-P suc n : Nat]]} = \interp{[[n]]};\mathsf{suc}$:
    \[
    \begin{array}{lll}
      \langle \id_{\interp{[[G]]}}, \interp{[[n]]};\mathsf{suc} \rangle;\mathsf{case}_{\interp{[[X]]}}
      &  = & \langle \id_{\interp{[[G]]}}, \interp{[[n]]} \rangle;(\id_{\interp{[[G]]}} \times \mathsf{suc});\mathsf{case}_{\interp{[[X]]}}\\
      &  = & \langle \id_{\interp{[[G]]}}, \interp{[[n]]} \rangle;\interp{[[t2]]}.
    \end{array}
    \]
  \end{itemize}
\end{proof}
% subsection the_programmatic_fragment (end)
% section the_fragments_and_their_semantics (end)

\section{A Termination/Non-Termination Adjoint Model}
\label{sec:a_termination/non-termination_adjoint_model}
We have so far laid out both the syntactic and categorical
formalizations of the logical fragment and the programmatic fragment.
At this point we turn to joining these two fragments together using an
adjunction, but before this we must acquaint ourselves with the basic
notions involved in the construction of such a model.

\subsection{The Basic Tools}
\label{subsec:the_basic_tools}
We have been working in cartesian closed categories, but much of the
basic categorical tools we will employ arose from the study of
symmetric monoidal closed categories which are at the heart of
categorical models of linear logic.  This section consists of
well-known categorical tools and is based on the Benton's
presentation; see \cite{?} for references on the history of these
notions.

A symmetric monoidal closed category, $(\cat{C}, \otimes, I, \limp,
\lambda, \rho, \beta, \alpha)$, consists of a bifunctor $\otimes :
\cat{C} \times \cat{C} \to \cat{C}$, called the tensor product of
$\cat{C}$, which is associative, there is a natural isomorphism,
$\alpha_{A,B,C} : (A \otimes B) \otimes C \to A \otimes (B \otimes
C)$, has a unit $I$, that is, there are natural isomorphisms
$\lambda_A : A \otimes I \to I$ and $\rho_A : I \otimes A \to A$, the
tensor product is symmetric and hence there is a natural isomorphism
$\beta_{A,B} : A \otimes B \to B \otimes A$, and finally, the category
is closed.  That is, the following is a natural bijection:
\[
\Hom{\cat{C}}{A \otimes B,C} \cong \Hom{\cat{C}}{A, B \limp C}
\]
Furthermore, the natural transformations in the definition of a
symmetric monoidal category are subject to several coherence
conditions.  For the complete definition see
Appendix~\ref{sec:symmetric_monoidal_closed_categories}.

A functor, $(F,\m{})$, between the two symmetric monoidal categories
$(\cat{M}_1, \otimes_1, I_1)$ and $(\cat{M}_2, \otimes_2, I_2)$ is an
ordinary functor $F : \cat{M}_1 \mto \cat{M}_2$ such that there is a
natural transformation $\m{A,B} : FA \otimes_2 FB \mto F(A \otimes_1
B)$ and a morphism $\m{I} : I_2 \mto FI_1$.  These are also subject to
some coherence conditions; see
Appendix~\ref{sec:symmetric_monoidal_closed_categories}.  Natural
transformations between symmetric monoidal functors, $(F, \m{})$ and
$(G, \n{})$ are ordinary natural transformations $f : F \to G$ subject
to the following coherence conditions:
\begin{mathpar}
  \scriptsize
    \bfig
    \square<1000,500>[
      FA \otimes_2 FB`
      F(A \otimes_1 B)`
      GA \otimes_2 GB`
      G(A \otimes_1 B);
      m_{A,B}`
      f_A \otimes_2 f_B`
      f_{A \otimes_1 B}`
      n_{A,B}]
    \efig
    \and
    \bfig
    \Vtriangle/->`<-`<-/[
      FI_1`
      GI_1`
      I_2;
      f_{I_1}`
      m_{I_1}`
      n_{I_1}]
    \efig
  \end{mathpar}  

We will be making heavy use of symmetric monoidal adjunctions, and so
we give their full definition.
\begin{definition}
  \label{def:SMCADJ}
  Suppose $(\cat{M}_1,I_1,\otimes_1)$ and $(\cat{M}_2,I_2,\otimes_2)$
  are SMCs, and $(F,m)$ is a symmetric monoidal functor between
  $\cat{M}_1$ and $\cat{M}_2$ and $(G,n)$ is a symmetric monoidal
  functor between $\cat{M}_2$ and $\cat{M}_1$.  Then a
  \textbf{symmetric monoidal adjunction} is an ordinary adjunction
  $\cat{M}_1 : F \dashv G : \cat{M}_2$ such that the unit,
  $\varepsilon : A \to GFA$, and the counit, $\eta_A : FGA \to A$, are
  symmetric monoidal natural transformations.  Thus, the following
  diagrams must commute:
  \begin{mathpar}
    \scriptsize
    \bfig
    \qtriangle|amm|<1000,500>[
      FGA \otimes_1 FGB`
      FG(A \otimes_1 B)`
      A \otimes_1 B;
      \q{A,B}`
      \eta_A \otimes_1 \eta_B`
      \eta_{A \otimes_1 B}]
    \efig
    \and
    \bfig
    \Vtriangle|amm|/->`<-`=/[
      FGI_1`
      I_1`
      I_1;
      \eta_{I_1}`
      \q{I_1}`]
    \efig
    \and
    \bfig
    \dtriangle|mmb|<1000,500>[
      A \otimes_2 B`
      GFA \otimes_2 GFB`
      GF(A \otimes_2 B);
      \varepsilon_A \otimes_2 \varepsilon_B`
      \varepsilon_{A \otimes_2 B}`
      \p{A,B}]
    \efig
    \and
    \bfig
    \Vtriangle|amm|/->`=`<-/[
      I_2`
      GFI_2`
      I_2;
      \varepsilon_{I_2}``
      p_{I_2}]
    \efig
  \end{mathpar}
  Note that $\p{}$ and $\q{}$ exist because $(FG,\q{})$ and
  $(GF,\p{})$ are symmetric monoidal functors.
\end{definition}

In addition, we make use of symmetric monoidal monads and comonads.
We give their definition next.
\begin{definition}
  \label{def:symm-monoidal-monad}
  A \textbf{symmetric monoidal monad} on a symmetric monoidal
  category $\cat{C}$ is a triple $(T,\eta, \mu)$, where
  $(T,\n{})$ is a symmetric monoidal endofunctor on $\cat{C}$,
  $\eta_A : A \mto TA$ and $\mu_A : T^2A \to TA$ are
  symmetric monoidal natural transformations, which make the following
  diagrams commute:
  \begin{mathpar}
    \scriptsize
    \bfig
    \square|ammb|<600,600>[
      T^3 A`
      T^2A`
      T^2A`
      TA;
      \mu_{TA}`
      T\mu_A`
      \mu_A`
      \mu_A]
    \efig
    \and
    \bfig
    \Atrianglepair/=`<-`=`->`<-/<600,600>[
      TA`
      TA`
      T^2 A`
      TA;`
      \mu_A``
      \eta_{TA}`
      T\eta_A]
    \efig
  \end{mathpar}
\end{definition}

\begin{definition}
  \label{def:symm-monoidal-comonad}
  A \textbf{symmetric monoidal comonad} on a symmetric monoidal
  category $\cat{C}$ is a triple $(T,\varepsilon, \delta)$, where
  $(T,\m{})$ is a symmetric monoidal endofunctor on $\cat{C}$,
  $\varepsilon_A : TA \mto A$ and $\delta_A : TA \to T^2 A$ are
  symmetric monoidal natural transformations, which make the following
  diagrams commute:
  \begin{mathpar}
    \scriptsize
    \bfig
    \square|amma|<600,600>[
      TA`
      T^2A`
      T^2A`
      T^3A;
      \delta_A`
      \delta_A`
      T\delta_A`
      \delta_{TA}]
    \efig
    \and
    \bfig
    \Atrianglepair/=`->`=`<-`->/<600,600>[
      TA`
      TA`
      T^2 A`
      TA;`
      \delta_A``
      \varepsilon_{TA}`
      T\varepsilon_A]
    \efig
  \end{mathpar}
\end{definition}
The use of symmetric monoidal natural transformations in the previous
two definitions packs a lot of punch.  See what these conditions look
like in Appendix~\ref{sec:symmetric_monoidal_closed_categories}.  
% subsection the_basic_tools (end)

Specializing the basic tools just presented to cartesian categories
has surprising implications especially when considering symmetric
monoidal adjunctions.

\subsection{Symmetric Monoidal Adjunctions on Cartesian Categories}
\label{subsec:symmetric_monoidal_adjunctions_on_cartesian_categories}
First, it is well-known that any cartesian category is a symmetric
monoidal category. The following lemma summarizes this structure.
\begin{lemma}[CCCs are SMCs]
  \label{lemma:CCC-is-SMC}
  Suppose $(\cat{C}, 1, \times, \Rightarrow)$ is a cartesian closed
  category.  Then the following defines the structure of a symmetric
  monoidal category:
  \[
  \begin{array}{rll}
    \lambda_A & = & \pi_2 : 1 \times A \mto A\\
    \lambda^{-1}_A & = & \langle \t_A , \id_A  \rangle : A \mto 1 \times A\\
    \\
    \rho_A & = & \pi_1 : A \times 1 \mto A\\
    \rho^{-1}_A & = & \langle \id_A , \t_A  \rangle : A \mto A \times 1\\
    \\
    \alpha_{A,B,C} & = & \langle \pi_1;\pi_1, \pi_2 \times \id_C \rangle : (A \times B) \times C \mto A \times (B \times C)\\
    \alpha^{-1}_{A,B,C} & = & \langle \id_A \times \pi_1, \pi_2;\pi_2 \rangle : A \times (B \times C) \mto (A \times B) \times C\\
    \\
    \beta_{A,B} & = & \langle \pi_2, \pi_1 \rangle : A \times B \mto B \times A\\
  \end{array}
  \]
  Each of the above morphisms satisfy the appropriate diagrams.
\end{lemma}

Symmetric monoidal functors can be seen as functors that essentially
preserve the monoidal structure, and this kind of preservation can be
given to other types of functors.  The following definition describes
how symmetric monoidal functors can be modified to preserve the
cartesian structure of a cartesian category.

\begin{definition}
  \label{def:prod-functor}
  A \textbf{product functor}, $(F,m) : \cat{C}_1 \to \cat{C}_2$,
  between two cartesian categories consists of an ordinary functor $F
  : \cat{C}_1 \to \cat{C}_2$, a natural transformation $m_{A,B} : FA
  \times FB \mto F(A \times B)$, and a map $m_1 : 1 \to F1$ subject to the
  following coherence diagrams:
  \begin{mathpar}
    \bfig
    \btriangle|mmm|<1000,500>[
      FA \times FB`
      F(A \times B)`
      FA;
      m_{A.B}`
      \pi_1`
      F\pi_1]
    \efig
    \and
    \bfig    
    \btriangle|mmm|<1000,500>[
      FA \times FB`
      F(A \times B)`
      FB;
      m_{A.B}`
      \pi_2`
      F\pi_2]
    \efig
    \and
    \bfig
    \btriangle|mmm|<1000,500>[
      FC`
      FA \times FB`
      F(A \times B);
      \langle Ff, Fg \rangle`
      F(\langle f, g \rangle)`
      m_{A,B}]
    \efig
    \and
    \bfig
    \btriangle|mmm|<1000,500>[
      FA`
      1`
      F1;
      \t_{FA}`
      F\t_{A}`
      m_1]
    \efig
  \end{mathpar} 
\end{definition}
\noindent
It is not surprising that product functors are symmetric monoidal functors.
\begin{lemma}[Product Functors are Symmetric Monoidal]
  \label{lemma:product_functors_are_symmetric_monoidal}
  If $(F,m) : \cat{C}_1 \to \cat{C}_2$ is a product functor betwen cartesian categories, then
  $(F,m)$ is a symmetric monoidal functor.
\end{lemma}
\iffalse
\begin{proof}
  We simply show that each diagram in the definition of symmetric
  monoidal functor commutes.  This proof requires the following facts
  about cartesian categories:
  \[
  \begin{array}{lll}
    f;\langle g , h \rangle = \langle f;g, f;g \rangle & \text{(P1)}\\
    \\
    f \times g = \langle \pi_1;f , \pi_2;g \rangle & \text{(P2)}\\
    \\
    \langle f , g \rangle;(h \times i) = \langle f;h, g;i \rangle & \text{(P3)}\\
    \\
    (f \times g);\pi_1 = \pi_1;f & \text{(P4)}\\
    \\
    (f \times g);\pi_2 = \pi_2;g & \text{(P5)}\\
  \end{array}
  \]
  \begin{itemize}
  \item[] Associativity:\ \\
    \[
    \bfig
    \vSquares|ammmmma|/->`->`->``->`->`->/[
      (FA \times FB) \times FC`
      FA \times (FB \times FC)`
      F(A \times B) \times FC`
      FA \times F(B \times C)`
      F((A \times B) \times C)`
      F(A \times (B \times C));
      {\alpha}_{FA,FB,FC}`
      m_{A,B} \times \id_{FC}`
      \id_{FA} \times m_{B,C}``
      m_{A \times B,C}`
      m_{A,B \times C}`
      F{\alpha}_{A,B,C}]
    \efig
    \]
    In this case it is easiest to show that each side of the diagram
    reduces to the same morphism.

    First, the right side reduces in the following way:
    \begin{center}
      \begin{math}
        \begin{array}{lll}
          \alpha_{FA,FB,FC};(\id_{FA} \times m_{B,C});m_{A,B \times C} \\
          \,\,\,\,\,\,\,\,= \langle \pi_1;\pi_1, \pi_2 \times \id_{FC} \rangle;(\id_{FA} \times m_{B,C});m_{A,B \times C} & \text{(Definition)}\\
          \,\,\,\,\,\,\,\,= \langle \pi_1;\pi_1, (\pi_2 \times \id_{FC});m_{B,C} \rangle;m_{A,B \times C} & \text{(P3)}\\
        \end{array}
      \end{math}
    \end{center}

    Now the left side reduces to the same morphism:
    \[
    \scriptsize
    \begin{array}{lll}
      (m_{A,B} \times \id_{FC});m_{A \times B,C};\underline{F\alpha_{A,B,C}} \\
    \,\,\,\,\,\,\,\,= (m_{A,B} \times \id_{FC});m_{A \times B,C};\underline{F(\langle \pi_1;\pi_1, \pi_2 \times \id_C \rangle)} & \text{(Definition)}\\
   \,\,\,\,\,\,\,\,= (m_{A,B} \times \id_{FC});\underline{m_{A \times B,C};\langle F\pi_1;F\pi_1, F(\pi_2 \times \id_C) \rangle;m_{A,B \times C}} & \text{(Product Functor)}\\
   \,\,\,\,\,\,\,\,= (m_{A,B} \times \id_{FC});\langle \underline{m_{A \times B,C};F\pi_1};F\pi_1, m_{A \times B,C};F(\pi_2 \times \id_C) \rangle;m_{A,B \times C} & \text{(P1)}\\
   \,\,\,\,\,\,\,\,= (m_{A,B} \times \id_{FC});\langle \pi_1;F\pi_1, m_{A \times B,C};\underline{F(\pi_2 \times \id_C)} \rangle;m_{A,B \times C} & \text{(Product Functor)}\\
   \,\,\,\,\,\,\,\,= (m_{A,B} \times \id_{FC});\langle \pi_1;F\pi_1, m_{A \times B,C};\underline{F(\langle \pi_1;\pi_2, \pi_2 \rangle)} \rangle;m_{A,B \times C} & \text{(P2)}\\
   \,\,\,\,\,\,\,\,= (m_{A,B} \times \id_{FC});\langle \pi_1;F\pi_1, \underline{m_{A \times B,C};\langle F\pi_1;F\pi_2, F\pi_2 \rangle};m_{B,C} \rangle;m_{A,B \times C} & \text{(Product Functor)}\\
   \,\,\,\,\,\,\,\,= (m_{A,B} \times \id_{FC});\langle \pi_1;F\pi_1, \langle \underline{m_{A \times B,C};F\pi_1};F\pi_2, \underline{m_{A \times B,C};F\pi_2} \rangle;m_{B,C} \rangle;m_{A,B \times C} & \text{(P1)}\\
   \,\,\,\,\,\,\,\,= (m_{A,B} \times \id_{FC});\langle \pi_1;F\pi_1, \underline{\langle \pi_1;F\pi_2, \pi_2 \rangle};m_{B,C} \rangle;m_{A,B \times C} & \text{(Product Functor)}\\
   \,\,\,\,\,\,\,\,= \underline{(m_{A,B} \times \id_{FC});\langle \pi_1;F\pi_1, (F\pi_2 \times \id_{FC});m_{B,C} \rangle};m_{A,B \times C} & \text{(P2)}\\
   \,\,\,\,\,\,\,\,= \langle \underline{(m_{A,B} \times \id_{FC});\pi_1};F\pi_1, (m_{A,B} \times \id_{FC});(F\pi_2 \times \id_{FC});m_{B,C} \rangle;m_{A,B \times C} & \text{(P1)}\\
   \,\,\,\,\,\,\,\,= \langle \pi_1;\underline{m_{A,B};F\pi_1}, (m_{A,B} \times \id_{FC});(F\pi_2 \times \id_{FC});m_{B,C} \rangle;m_{A,B \times C} & \text{(P4)}\\
   \,\,\,\,\,\,\,\,= \langle \pi_1;\pi_1, \underline{(m_{A,B} \times \id_{FC});(F\pi_2 \times \id_{FC})};m_{B,C} \rangle;m_{A,B \times C} & \text{(Product Functor)}\\
   \,\,\,\,\,\,\,\,= \langle \pi_1;\pi_1, (\underline{(m_{A,B};F\pi_2)} \times \id_{FC});m_{B,C} \rangle;m_{A,B \times C} & \text{(Functor)}\\
   \,\,\,\,\,\,\,\,= \langle \pi_1;\pi_1, (\pi_2 \times \id_{FC});m_{B,C} \rangle;m_{A,B \times C} & \text{(Product Functor)}\\
    \end{array}
    \]

  \item[] Left Unitor:\\
    \[
    \bfig
    \square|amma|/->`->`<-`->/<1000,500>[
      1 \times FA`
      FA`
      F1 \times FA`
      F(1 \times A);
      {\lambda}_{FA}`
      m_{1} \times \id_{FA}`
      F{\lambda}_{A}`
      m_{1,A}]
    \efig
    \]
  
  First, notice that $\lambda_{FA} = \pi_2$. We can show the following:
  \[
  \begin{array}{lll}
    (m_1 \times \id_{FA});m_{1,A};F\lambda_A\\
    \,\,\,\,\,\,\,\,= (m_1 \times \id_{FA});m_{1,A};F\pi_2 & \text{(Definition)}\\
    \,\,\,\,\,\,\,\,= (m_1 \times \id_{FA});\pi_2 & \text{(Product Functor)}\\
    \,\,\,\,\,\,\,\,= \pi_2 & \text{(P5)}\\
  \end{array}
  \]

\item[] Right Unitor:\\
  \[
  \bfig
  \square|amma|/->`->`<-`->/<1000,500>[
    FA \times 1`
    FA`
    FA \times F1`
    F(A \times 1);
    {\rho}_{FA}`
    \id_{FA} \times m_{1}`
    F{\rho}_{A}`
    m_{A,1}]
  \efig
  \]
  \end{itemize}
  This case is similar to the previous one:
  \[
  \begin{array}{lll}
    (\id_{FA} \times m_1);m_{A,1};F\rho_A\\
    \,\,\,\,\,\,\,\,=  (\id_{FA} \times m_1);m_{A,1};F\pi_1 & \text{(Definition)}\\
    \,\,\,\,\,\,\,\,=  (\id_{FA} \times m_1);\pi_1 & \text{(Product Functor)}\\
    \,\,\,\,\,\,\,\,=  \pi_1 & \text{(P4)}\\
    \,\,\,\,\,\,\,\,=  \rho_{FA} & \text{(Definition)}\\
  \end{array}
  \]

\item[] Symmetry:\\
  \[
  \bfig
  \square|amma|/->`->`->`->/<1000,500>[
    FA \times FB`
    FB \times FA`
    F(A \times B)`
    F(B \times A);
    {\beta}_{FA,FB}`
    m_{A,B}`
    m_{B,A}`
    F{\beta}_{A,B}]
  \efig
  \]
  First, notice that $\beta_{A,B} = \langle \pi_2, \pi_1 \rangle$.  Thus, we have the following:
  \[
  \begin{array}{lll}
    \beta_{FA,FB};m_{B,A}\\
    \,\,\,\,\,\,\,\,=  \langle \pi_2, \pi_1 \rangle;m_{B,A} & \text{(Definition)}\\
    \,\,\,\,\,\,\,\,=  \langle m_{A,B};F\pi_2, m_{A,B};F\pi_1 \rangle;m_{B,A} & \text{(Product Functor)}\\
    \,\,\,\,\,\,\,\,=  m_{A,B};\langle F\pi_2, F\pi_1 \rangle;m_{B,A} & \text{(P1)}\\
    \,\,\,\,\,\,\,\,=  m_{A,B};F(\langle \pi_2, \pi_1 \rangle) & \text{(Product Functor)}\\
    \,\,\,\,\,\,\,\,=  m_{A,B};F\beta_{A,B} & \text{(Definition)}\\
  \end{array}
  \]
\end{proof}
\fi
\noindent
A symmetric monoidal functor, $(F, \m{})$, is called \textbf{strong}
if $\m{A,B} : FA \otimes_2 FB \mto F(A \otimes_1 B)$ and $\m{I} : I_2
\mto FI_1$ are isomorphisms.  It turns out that product functors are
in fact strong.
\begin{lemma}[Product Functors Isomorphisms]
  \label{lemma:product_functors_iso}
  If $(F,m) : \cat{C}_1 \mto \cat{C}_2$ is a product functor, then
  $m_{A,B} : FA \times FB \mto F(A \times B)$ and $m_1 : 1 \mto F1$ are
  isomorphisms.
\end{lemma}
\iffalse
\begin{proof}
  Define $m^{-1}_1 = \t_{F 1} : F1 \mto 1$.  Then we have the following:
  \[
  \begin{array}{lll}
    m_1;t_{F1} = \id_{1} & \text{(Uniqueness)}\\    
  \end{array}
  \]
  and
  \[
  \begin{array}{llll}
    t_{F1};m_1
    & = & F\t_t & \text{(Product Functor)}\\
    & = & F\id_t & \text{(Uniqueness)}\\
    & = & \id_{Ft}     
  \end{array}
  \]

  Now define $m^{-1}_{A,B} = \langle F\pi_1, F\pi_2 \rangle$.  Then we have the following:
  \[
  \begin{array}{llll}
    m_{A,B};m^{-1}_{A,B}
    & = & m_{A,B};\langle F\pi_1, F\pi_2 \rangle & \text{(Definition)}\\
    & = & \langle m_{A,B};F\pi_1, m_{A,B};F\pi_2 \rangle & \text{(P1)}\\
    & = & \langle \pi_1, \pi_2 \rangle & \text{(Product Functor)}\\
    & = & \id_{FA \times FB} & \text{(Cartesian)}\\
  \end{array}
  \]
  and
  \[
  \begin{array}{llll}
    m^{-1}_{A,B};m_{A,B}
    & = & \langle F\pi_1, F\pi_2 \rangle;m_{A,B} & \text{(Definition)}\\
    & = & F(\langle \pi_1, \pi_2 \rangle) & \text{(Product Functor)}\\
    & = & F(\id_{A \times B}) & \text{(Cartesian)}\\
    & = & \id_{F(A \times B)}\\    
  \end{array}
  \]
\end{proof}
\fi
\noindent
Surprisingly, strong symmetric monoidal functors between cartesian
categories preserve the cartesian structure, and hence, are in fact
product functors.  
\begin{lemma}[Strong Symmetric Monoidal Functors are Product Functors]
  \label{lemma:strong-symmetric_monoidal_functors_are_product_functors}
  If $(F,m) : \cat{C}_1 \mto \cat{C}_2$ is a strong symmetric monoidal
  functor between cartesian closed categories, then $(F,m)$ is a
  product functor.
\end{lemma}
\iffalse
\begin{proof}
  \ \\
  \begin{itemize}
  \item[] Case 1.\\
    \[
      \bfig
      \btriangle|mmm|<1000,500>[
        FA`
        1`
        F1;
        \t_{FA}`
        F\t_{A}`
        m_1]
      \efig
    \]
    This case follows because it must be the case that
    \[F\t_{A};\m{1}^{-1} = \t_{FA} : FA \mto 1\]
    by the uniqueness of the terminal arrow.
    
  \item[] Case 2.\\
    \[
      \bfig
      \btriangle|mmm|<1000,500>[
        FA \times FB`
        F(A \times B)`
        FA;
        m_{A,B}`
        \pi_1`
        F\pi_1]
      \efig
      \]
      First, $\rho_A = \pi_1 : A \times 1 \mto A$, and hence, we have the following:
      \[
      \begin{array}{lll}
        (\id_A \times \t_A);\rho_1 & = & \langle \pi_1;\id_A, \pi_2;\t_A \rangle;\rho_1\\
        & = & \langle \pi_1;\id_A, \pi_2;\t_A \rangle;\pi_1\\
        & = & \pi_1 : A \times B \mto A
      \end{array}
      \]
      This case follows by the following equational reasoning:
      \[
      \begin{array}{llllllll}
        \m{A,B};F\pi_1
        & = & \m{A,B};F(\id_A \times \t_B);F\rho_A & \text{(Above Equation)}\\
        & = & (\id_{FA} \times F(\t_{B}));\m{A,1};F\rho_A & \text{(Naturality of $\m{}$)}\\
        & = & (\id_{FA} \times \t_{FB};\m{1});\m{A,1};F\rho_A & \text{(Case 1)}\\
        & = & \langle \id_{FA} \times \t_{FB} \rangle;(\id_{FA} \times \m{1});\m{A,1};F\rho_A & \text{(Cartesian)}\\
        & = & \langle \id_{FA} \times \t_{FB} \rangle;\rho_{FA} & \text{($F$ is Symmetric Monoidal)}\\
        & = & \langle \id_{FA} \times \t_{FB} \rangle;\pi_1 & \text{(Cartesian)}\\
        & = & \pi_1 & \text{(Cartesian)}\\ 
      \end{array}
      \]
    
  \item[] Case 3.\\    
    \[
      \bfig    
      \btriangle|mmm|<1000,500>[
        FA \times FB`
        F(A \times B)`
        FB;
        m_{A,B}`
        \pi_2`
        F\pi_2]
      \efig
    \]
    First, $\lambda_A = \pi_2 : 1 \times A \mto A$, and hence, we have the following:
      \[
      \begin{array}{lll}
        (\t_A \times \id_A);\lambda_A & = & \langle \pi_1;\t_A, \pi_2;\id_A \rangle;\lambda_A\\
        & = & \langle \pi_1;\t_A, \pi_2;\id_A \rangle;\pi_2\\
        & = & \pi_2 : A \times B \mto B
      \end{array}
      \]
      This case follows from similar equational reasoning
      as the previous case.      
      
  \item[] Case 4.\\    
    \[
    \bfig
    \btriangle|mmm|<1000,500>[
      FC`
      FA \times FB`
      F(A \times B);
      \langle Ff, Fg \rangle`
      F(\langle f, g \rangle)`
      m_{A,B}]
    \efig
    \]
    First, note that the following diagrams commute:
    \[
    \bfig
    \Atrianglepair|mmmaa|/->`->`->`<-`->/<1000,600>[
      FA \times FB`
      FA`
      F(A \times B)`
      FB;
      \pi_1`
      \m{A,B}`
      \pi_2`
      F\pi_1`
      F\pi_2]
    \morphism(1000,1200)|m|<0,-600>[FC`FA \times FB;\langle Ff, Fg \rangle]
    \morphism(1000,1200)/{@{>}@/^-1em/}/<-1000,-1200>[FC`FA;Ff]
    \morphism(1000,1200)/{@{>}@/^1em/}/<1000,-1200>[FC`FB;Fg]
    \place(600,600)[1]
    \place(1400,600)[2]
    \place(750,250)[3]
    \place(1230,250)[4]
    \efig
    \]
    Diagrams 1 and 2 hold by the universal mapping property for the
    cartesian product, and diagrams three and four hold by cases two
    and three of this proof.

    We now have the following two bits of equational reasoning:
    \[
    \begin{array}{lllll}
      F(\langle f,g \rangle);\m{A,B}^{-1};\pi_1
      & = & F(\langle f,g \rangle);\m{A,B}^{-1};\m{A,B};F\pi_1 & \text{(Diagram 3)}\\
      & = & F(\langle f,g \rangle);F\pi_1 & \text{}\\
      & = & F(\langle f,g \rangle;\pi_1) & \text{}\\
      & = & Ff & \text{(Cartesian)}\\
    \end{array}      
    \]
    and
    \[
    \begin{array}{lllll}
      F(\langle f,g \rangle);\m{A,B}^{-1};\pi_2
      & = & F(\langle f,g \rangle);\m{A,B}^{-1};\m{A,B};F\pi_2 & \text{(Diagram 4)}\\
      & = & F(\langle f,g \rangle);F\pi_2 & \text{}\\
      & = & F(\langle f,g \rangle;\pi_2) & \text{}\\
      & = & Fg & \text{(Cartesian)}\\
    \end{array}      
    \]
    Thus, by the uniqueness of $\langle Ff,Fg \rangle$, due to the
    universal mapping property of the cartesian product, it must be
    the case that the following holds:
    \[
    F(\langle f,g \rangle);\m{A,B}^{-1} = \langle Ff,Fg \rangle
    \]
    Thus, $F(\langle f,g \rangle) = \langle Ff,Fg \rangle;\m{A,B}$.
  \end{itemize}  
\end{proof}
\fi

The following lemma shows that the left adjoint in any symmetric
monoidal adjunction is strong, and hence, when both categories in the
adjunction are cartesian we know the left adjoint preserves the
cartesian structure.  This was first shown by Benton but for LNL
models where one of the categories is cartesian closed, and thus, the
following result is a generalization \cite{benton1995}.

\begin{lemma}[Symmetric Monoidal Adjunctions have Strong Left-adjoints]
  \label{lemma:product_functors_strong}
  If $(F,m) \dashv (G,n) : \cat{C}_1 \mto \cat{C}_2$ is any symmetric
  monoidal adjunction, then $F$ is strong.
\end{lemma}
\begin{proof}
  In \cite{benton1995} Benton shows that this result holds for an LNL
  model, but his exact proof works for the case when both categories
  in the adjunction are symmetric monoidal.  
\end{proof}

If we have a symmetric monoidal adjunction between cartesian
categories we know from the previous result that the left adjoint is a
product functor, but it turns out that we also know the right adjoint
must be a product functor as well, because right adjoints preserve
limits.

\begin{lemma}[Symmetric Monoidal Adjunctions and Product Functors]
  \label{lemma:adjoint-CCC-prod-func}
  Suppose $(F,m) : \cat{C}_1 \to \cat{C}_2$ and $(G,n) : \cat{C}_2 \to
  \cat{C}_1$ are symmetric monoidal functors between cartesian-closed
  categories, and $F \vdash G$ is a symmetric monoidal adjunction.
  Then $(F,m)$ and $(G,n)$ are product functors.
\end{lemma}
\begin{proof}
  We know from Lemma~\ref{lemma:product_functors_strong} $(F,m)$ is
  strong, and hence, by
  Lemma~\ref{lemma:strong-symmetric_monoidal_functors_are_product_functors}
  it is a product functor.  Now we must show that $(G,n)$ is strong,
  but it is a right adjoint, and hence, preserves all limits.  Thus,
  it is strong.
\end{proof}
\noindent
The remainder of the paper will deal specifically with product
functors due to the last result.
% subsection symmetric_monoidal_adjunctions_on_cartesian_categories (end)
% section a_termination/non-termination_adjoint_model (end)


\section{Related Work}
\label{sec:related_work}
\todo[inline]{Write}
% section related_work (end)

\section{Conclusion}
\label{sec:conclusion}
\todo[inline]{Write}
% section conclusion (end)

\bibliographystyle{plain}
\bibliography{ref}

\nocite{*}
\appendix

\section{Symmetric Monoidal Closed Categories}
\label{sec:symmetric_monoidal_closed_categories}

\begin{definition}
  \label{def:monoidal-category}
  A \textbf{symmetric monoidal category (SMC)} is a category, $\cat{M}$,
  with the following data:
  \begin{itemize}
  \item An object $I$ of $\cat{M}$,
  \item A bi-functor $\otimes : \cat{M} \times \cat{M} \mto \cat{M}$,
  \item The following natural isomorphisms:
    \[
    \begin{array}{lll}
      \lambda_A : I \otimes A \mto A\\
      \rho_A : A \otimes I \mto A\\      
      \alpha_{A,B,C} : (A \otimes B) \otimes C \mto A \otimes (B \otimes C)\\
    \end{array}
    \]
  \item A natural transformation:
    \[
    \beta_{A,B} : A \otimes B \mto B \otimes A
    \]
  \item Subject to the following coherence diagrams:
    \begin{mathpar}
      \bfig
      \vSquares|ammmmma|/->`->```->``<-/[
        ((A \otimes B) \otimes C) \otimes D`
        (A \otimes (B \otimes C)) \otimes D`
        (A \otimes B) \otimes (C \otimes D)``
        A \otimes (B \otimes (C \otimes D))`
        A \otimes ((B \otimes C) \otimes D);
        \alpha_{A,B,C} \otimes \id_D`
        \alpha_{A \otimes B,C,D}```
        \alpha_{A,B,C \otimes D}``
        \id_A \otimes \alpha_{B,C,D}]      
      
      \morphism(1433,1000)|m|<0,-1000>[
        (A \otimes (B \otimes C)) \otimes D`
        A \otimes ((B \otimes C) \otimes D);
        \alpha_{A,B \otimes C,D}]
      \efig
      \and
      \bfig
      \hSquares|aammmaa|/->`->`->``->`->`->/[
        (A \otimes B) \otimes C`
        A \otimes (B \otimes C)`
        (B \otimes C) \otimes A`
        (B \otimes A) \otimes C`
        B \otimes (A \otimes C)`
        B \otimes (C \otimes A);
        \alpha_{A,B,C}`
        \beta_{A,B \otimes C}`
        \beta_{A,B} \otimes \id_C``
        \alpha_{B,C,A}`
        \alpha_{B,A,C}`
        \id_B \otimes \beta_{A,C}]
      \efig      
    \end{mathpar}
    \begin{mathpar}
      \bfig
      \Vtriangle[
        (A \otimes I) \otimes B`
        A \otimes (I \otimes B)`
        A \otimes B;
        \alpha_{A,I,B}`
        \rho_{A}`
        \lambda_{B}]
      \efig
      \and
      \bfig
      \btriangle[
        A \otimes B`
        B \otimes A`
        A \otimes B;
        \beta_{A,B}`
        \id_{A \otimes B}`
        \beta_{B,A}]
      \efig
      \and
      \bfig
      \Vtriangle[
        I \otimes A`
        A \otimes I`
        A;
        \beta_{I,A}`
        \lambda_A`
        \rho_A]
      \efig
    \end{mathpar}    
  \end{itemize}
\end{definition}

\begin{definition}
  \label{def:SMCC}
  A \textbf{symmetric monoidal closed category (SMCC)} is a symmetric
  monoidal category, $(\cat{M},I,\otimes)$, such that, for any object
  $B$ of $\cat{M}$, the functor $- \otimes B : \cat{M} \mto \cat{M}$
  has a specified right adjoint.  Hence, for any objects $A$ and $C$
  of $\cat{M}$ there is an object $A \limp B$ of $\cat{M}$ and a
  natural bijection:
  \[
  \Hom{\cat{M}}{A \otimes B}{C} \cong \Hom{\cat{M}}{A}{B \limp C}
  \]
\end{definition}

\begin{definition}
  \label{def:SMCFUN}
  Suppose $(\cat{M}_1,I_1,\otimes_1,\alpha_1,\lambda_1,\rho_1,\beta_1)$ and\\
  $(\cat{M}_2,I_2,\otimes_2,\alpha_2,\lambda_2,\rho_2,\beta_2)$ are SMCs.  Then a
  \textbf{symmetric monoidal functor} is a functor $F : \cat{M}_1 \mto
  \cat{M}_2$, a map $m_I : I_2 \mto FI_1$ and a natural transformation
  $m_{A,B} : FA \otimes_2 FB \mto F(A \otimes_1 B)$ subject to the
  following coherence conditions:
  \begin{mathpar}
    \bfig
    \vSquares|ammmmma|/->`->`->``->`->`->/[
      (FA \otimes_2 FB) \otimes_2 FC`
      FA \otimes_2 (FB \otimes_2 FC)`
      F(A \otimes_1 B) \otimes_2 FC`
      FA \otimes_2 F(B \otimes_1 C)`
      F((A \otimes_1 B) \otimes_1 C)`
      F(A \otimes_1 (B \otimes_1 C));
      {\alpha_2}_{FA,FB,FC}`
      m_{A,B} \otimes \id_{FC}`
      \id_{FA} \otimes m_{B,C}``
      m_{A \otimes_1 B,C}`
      m_{A,B \otimes_1 C}`
      F{\alpha_1}_{A,B,C}]
    \efig
    \and
    \bfig
    \square|amma|/->`->`<-`->/<1000,500>[
      I_2 \otimes_2 FA`
      FA`
      FI_1 \otimes_2 FA`
      F(I_1 \otimes_1 A);
      {\lambda_2}_{FA}`
      m_{I} \otimes \id_{FA}`
      F{\lambda_1}_{A}`
      m_{I_1,A}]
    \efig
    \and
    \bfig
    \square|amma|/->`->`<-`->/<1000,500>[
      FA \otimes_2 I_2`
      FA`
      FA \otimes_2 FI_1`
      F(A \otimes_1 I_1);
      {\rho_2}_{FA}`
      \id_{FA} \otimes m_{I}`
      F{\rho_1}_{A}`
      m_{A,I_1}]
    \efig
    \and
    \bfig
    \square|amma|/->`->`->`->/<1000,500>[
      FA \otimes_2 FB`
      FB \otimes_2 FA`
      F(A \otimes_1 B)`
      F(B \otimes_1 A);
      {\beta_2}_{FA,FB}`
      m_{A,B}`
      m_{B,A}`
      F{\beta_1}_{A,B}]
    \efig
  \end{mathpar}
\end{definition}

\begin{definition}
  \label{def:SMCNAT}
  Suppose $(\cat{M}_1,I_1,\otimes_1)$ and $(\cat{M}_2,I_2,\otimes_2)$
  are SMCs, and $(F,m)$ and $(G,n)$ are a symmetric monoidal functors
  between $\cat{M}_1$ and $\cat{M}_2$.  Then a \textbf{symmetric
    monoidal natural transformation} is a natural transformation,
  $f : F \mto G$, subject to the following coherence diagrams:
  \begin{mathpar}
    \bfig
    \square<1000,500>[
      FA \otimes_2 FB`
      F(A \otimes_1 B)`
      GA \otimes_2 GB`
      G(A \otimes_1 B);
      m_{A,B}`
      f_A \otimes_2 f_B`
      f_{A \otimes_1 B}`
      n_{A,B}]
    \efig
    \and
    \bfig
    \Vtriangle/->`<-`<-/[
      FI_1`
      GI_1`
      I_2;
      f_{I_1}`
      m_{I_1}`
      n_{I_1}]
    \efig
  \end{mathpar}  
\end{definition}

\begin{definition}
  \label{def:SMCADJ}
  Suppose $(\cat{M}_1,I_1,\otimes_1)$ and $(\cat{M}_2,I_2,\otimes_2)$
  are SMCs, and $(F,m)$ is a symmetric monoidal functor between
  $\cat{M}_1$ and $\cat{M}_2$ and $(G,n)$ is a symmetric monoidal
  functor between $\cat{M}_2$ and $\cat{M}_1$.  Then a
  \textbf{symmetric monoidal adjunction} is an ordinary adjunction
  $\cat{M}_1 : F \dashv G : \cat{M}_2$ such that the unit,
  $\varepsilon : A \to GFA$, and the counit, $\eta_A : FGA \to A$, are
  symmetric monoidal natural transformations.  Thus, the following
  diagrams must commute:
  \begin{mathpar}
    \bfig
    \qtriangle|amm|<1000,500>[
      FGA \otimes_1 FGB`
      FG(A \otimes_1 B)`
      A \otimes_1 B;
      \q{A,B}`
      \eta_A \otimes_1 \eta_B`
      \eta_{A \otimes_1 B}]
    \efig
    \and
    \bfig
    \Vtriangle|amm|/->`<-`=/[
      FGI_1`
      I_1`
      I_1;
      \eta_{I_1}`
      \q{I_1}`]
    \efig
    \and
    \bfig
    \dtriangle|mmb|<1000,500>[
      A \otimes_2 B`
      GFA \otimes_2 GFB`
      GF(A \otimes_2 B);
      \varepsilon_A \otimes_2 \varepsilon_B`
      \varepsilon_{A \otimes_2 B}`
      \p{A,B}]
    \efig
    \and
    \bfig
    \Vtriangle|amm|/->`=`<-/[
      I_2`
      GFI_2`
      I_2;
      \varepsilon_{I_2}``
      p_{I_2}]
    \efig
  \end{mathpar}
  Note that $\p{}$ and $\q{}$ exist because $(FG,\q{})$ and
  $(GF,\p{})$ are symmetric monoidal functors.
\end{definition}

\begin{definition}
  \label{def:symm-monoidal-monad}
  A \textbf{symmetric monoidal monad} on a symmetric monoidal
  category $\cat{C}$ is a triple $(T,\eta, \mu)$, where
  $(T,\n{})$ is a symmetric monoidal endofunctor on $\cat{C}$,
  $\eta_A : A \mto TA$ and $\mu_A : T^2A \to TA$ are
  symmetric monoidal natural transformations, which make the following
  diagrams commute:
  \begin{mathpar}
    \bfig
    \square|ammb|<600,600>[
      T^3 A`
      T^2A`
      T^2A`
      TA;
      \mu_{TA}`
      T\mu_A`
      \mu_A`
      \mu_A]
    \efig
    \and
    \bfig
    \Atrianglepair/=`<-`=`->`<-/<600,600>[
      TA`
      TA`
      T^2 A`
      TA;`
      \mu_A``
      \eta_{TA}`
      T\eta_A]
    \efig
  \end{mathpar}
  The assumption that $\eta$ and $\mu$ are symmetric
  monoidal natural transformations amount to the following diagrams
  commuting:
  \begin{mathpar}
    \bfig
    \dtriangle|mmb|<1000,600>[
      A \otimes B`
      TA \otimes TB`
      T(A \otimes B);
      \eta_A \otimes \eta_B`
      \eta_A`
      \n{A,B}]    
    \efig
    \and
    \bfig
    \Vtriangle/->`=`<-/<600,600>[
      I`
      TI`
      I;
      \eta_I``
      \n{I}]
    \efig
  \end{mathpar}
  \begin{mathpar}
    \bfig
    \square|ammm|/->`->``/<1050,600>[
      T^2 A \otimes T^2 B`
      T(TA \otimes TB)`
      TA \otimes TB`;
      \n{TA,TB}`
      \mu_A \otimes \mu_B``]

    \square(850,0)|ammm|/->``->`/<1050,600>[
      T(TA \otimes TB)`
      T^2(A \otimes B)``
      T(A \otimes B);
      T\n{A,B}``
      \mu_{A \otimes B}`]
    \morphism(-200,0)<2100,0>[TA \otimes TB`T(A \otimes B);\n{A,B}]
    \efig
    \and
    \bfig
    \square|ammb|/->`->`->`<-/<600,600>[
      I`
      TI`
      TI`
      TI;
      \n{I}`
      \n{T}`
      T\n{I}`
      \mu_I]
    \efig
  \end{mathpar}
\end{definition}

\begin{definition}
  \label{def:symm-monoidal-comonad}
  A \textbf{symmetric monoidal comonad} on a symmetric monoidal
  category $\cat{C}$ is a triple $(T,\varepsilon, \delta)$, where
  $(T,\m{})$ is a symmetric monoidal endofunctor on $\cat{C}$,
  $\varepsilon_A : TA \mto A$ and $\delta_A : TA \to T^2 A$ are
  symmetric monoidal natural transformations, which make the following
  diagrams commute:
  \begin{mathpar}
    \bfig
    \square|amma|<600,600>[
      TA`
      T^2A`
      T^2A`
      T^3A;
      \delta_A`
      \delta_A`
      T\delta_A`
      \delta_{TA}]
    \efig
    \and
    \bfig
    \Atrianglepair/=`->`=`<-`->/<600,600>[
      TA`
      TA`
      T^2 A`
      TA;`
      \delta_A``
      \varepsilon_{TA}`
      T\varepsilon_A]
    \efig
  \end{mathpar}
  The assumption that $\varepsilon$ and $\delta$ are symmetric
  monoidal natural transformations amount to the following diagrams
  commuting:
  \begin{mathpar}
    \bfig
    \qtriangle|amm|/->`->`->/<1000,600>[
      TA \otimes TB`
      T(A \otimes B)`
      A \otimes B;
      \m{A,B}`
      \varepsilon_A \otimes \varepsilon_B`
    \varepsilon_{A \otimes B}]
    \efig
    \and
    \bfig
    \Vtriangle|amm|/->`<-`=/<600,600>[
      TI`
      I`
      I;
      \m{I}`
      \varepsilon_I`]
    \efig    
  \end{mathpar}
  \begin{mathpar}
    \bfig
    \square|amab|/`->``->/<1050,600>[
      TA \otimes TB``
      T^2A \otimes T^2B`
      T(TA \otimes TB);`
      \delta_A \otimes \delta_B``
      \m{TA,TB}]
    \square(1050,0)|mmmb|/``->`->/<1050,600>[`
      T(A \otimes B)`
      T(TA \otimes TB)`
      T^2(A \otimes B);``
      \delta_{A \otimes B}`
      T\m{A,B}]
    \morphism(0,600)<2100,0>[TA \otimes TB`T(A \otimes B);\m{A,B}]
    \efig
    \and
    \bfig
    \square<600,600>[
      I`
      TI`
      TI`
      T^2I;
      \m{I}`
      \m{I}`
      \delta_I`
      T\m{I}]
    \efig
  \end{mathpar}
\end{definition}
% section symmetric_monoidal_categories (end)


%%% Local Variables: 
%%% mode: latex
%%% TeX-master: t
%%% End: 
