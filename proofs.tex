i\section{Proofs}
\label{sec:proofs}

\subsection{Proof of Soundness (Theorem~\ref{theorem:P-soundness})}
\label{subsec:proof_of_soundness}
We will prove part two explicitly which depends on part one in such
away that proving part one and then part two would require a lot of
repeated constructions.  We proceed by induction on the form of the
assumed term-in-context derivation, but since the majority of the
programmatic fragment is well-known to have a sound and complete
interpretation into a cartesian closed category we only show the
one of the cases for natural numbers.

First, recall that $\beta$-equality can be defined as follows in the
model -- for any morphisms $f : B \times C \mto D$, $g : A \mto C$,
and $h : A \to B$:
\[
\begin{array}{lllll}
  \langle h;\cur{f},g \rangle; \app_{B,C} & = &  \langle h,g \rangle;(\cur{f} \times \id_C);\app_{B,C}\\
  & = & \langle h, g\rangle;f\\
\end{array}
\]

Using $\beta$-equality we can show what we call $[[fix]]$-equality --
for any morphisms $f : A \to B$ and $g : B \to C$:
\[
\begin{array}{llll}
  \langle \cur{\pi_2;g},f \rangle;\app_{B,C}
  & = & \langle \id_A, f \rangle;\pi_2;g\\
  & = & f;g\\
\end{array}
\]

\begin{itemize}
\item[] Case.\ \\ 
  \begin{center}
    \begin{math}
      $$\mprset{flushleft}
      \inferrule* [right=\ifrName{Fix}] {
        [[D |-P t : X -> X]]
      }{[[D |-P fix t = t (fix t) : X]]}
    \end{math}
  \end{center}
  By part one we know there is a morphism:
  \[ \interp{[[D]]} \mto^{\interp{t}} \interp{[[X]]} \to \interp{[[X]]} \]
  of $\cat{P}$.  In addition, since any $\cat{P}$-model contains
  fixpoints we know there exists a morphism:
  \[
  (\interp{[[X]]} \to \interp{[[X]]}) \mto^{[[fix]]_{\interp{[[X]]}}} \interp{[[X]]}
  \]
  The interpretation of $[[D |-P fix : (X -> X) -> X]]$ is as follows:
  \[
  \interp{[[D]]} \mto^{\interp{[[D |-P fix : (X -> X) -> X]]} = \cur{\pi_2;[[fix]]_{\interp{[[X]]}}}} (\interp{[[X]]} \to \interp{[[X]]}) \to \interp{[[X]]}
  \]
  We obtain our result by the following equational reasoning:
  \[
  \scriptsize
  \begin{array}{lllll}
    \interp{[[D |-P fix t : X]]}\\
    \,\,\,\,\,\,= \langle \cur{\pi_2;[[fix]]_{\interp{[[X]]}}}, \interp{[[t]]} \rangle;\app_{\interp{[[X]]} \to \interp{[[X]]},\interp{[[X]]}}   & \text{(Definition)}\\
    \,\,\,\,\,\,= \interp{[[t]]};[[fix]]_{\interp{[[X]]}} & \text{($[[fix]]$-equality)}\\
    \,\,\,\,\,\,= \interp{[[t]]};\langle \id_{\interp{[[X]]} \to \interp{[[X]]}}, [[fix]]_{\interp{[[X]]}} \rangle;\app_{\interp{[[X]]},\interp{[[X]]}} & \text{(Fixpoint)}\\
    \,\,\,\,\,\,= \langle \interp{[[t]]}, \interp{[[t]]};[[fix]]_{\interp{[[X]]}} \rangle;\app_{\interp{[[X]]},\interp{[[X]]}} & \text{(Cartesian)}\\
    \,\,\,\,\,\,= \langle \interp{[[t]]}, \langle \cur{\pi_2;[[fix]]_{\interp{[[X]]}}},\interp{[[t]]} \rangle;\app_{\interp{[[X]]} \to \interp{[[X]]},\interp{[[X]]}} \rangle;\app_{\interp{[[X]]},\interp{[[X]]}} & \text{($[[fix]]$-equality)}\\
    \,\,\,\,\,\,= \langle \interp{[[t]]}, \langle \interp{[[D |-P fix : (X -> X) -> X]]},\interp{[[t]]} \rangle;\app_{\interp{[[X]]} \to \interp{[[X]]},\interp{[[X]]}} \rangle;\app_{\interp{[[X]]},\interp{[[X]]}} & \text{(Definition)}\\
    \,\,\,\,\,\,= \langle \interp{[[t]]}, \interp{[[D |-P fix t : X]]} \rangle;\app_{\interp{[[X]]},\interp{[[X]]}} & \text{(Definition)}\\
    \,\,\,\,\,\,= \interp{[[D |-P t (fix t) : X]]} & \text{(Definition)}\\      
  \end{array}
  \]

\item[] Case.\ \\ 
  \begin{center}
    \begin{math}
      $$\mprset{flushleft}
      \inferrule* [right=\ifrName{CaseB}] {
        [[D |-P t1 : X && D, x : Nat |-P t2 : X]]
      }{[[D |-P case 0 {0 -> t1, suc x -> t2} = t1 : X]]}
    \end{math}
  \end{center}
  We have the following morphisms by part one:
  \[
  \begin{array}{c}
    \interp{[[D]]} \mto^{\interp{[[t1]]}} \interp{[[X]]}\\
    \\
    \interp{[[D]]} \times [[Nat]] \mto^{\interp{[[t2]]}} \interp{[[X]]}
  \end{array}
  \]
  Then by the definition of SNNO we know there exists an unique
  morphism
  \[ [[Y x Nat]] \mto^{\mathsf{case}_{\interp{[[X]]}}} \interp{[[X]]}\]
  such that the following is one equation that holds:
  \[
  \begin{array}{lll}
    (\id_{\interp{[[D]]}} \times \mathsf{z});\mathsf{case}_{\interp{[[X]]}} = \pi_1;\interp{[[t1]]}
  \end{array}
  \]
  
  We also have the following interpretation for any $\interp{[[D |-P t : Nat]]}$:
  \[
  \interp{[[case t {0 -> t1, suc x -> t2}]]} =
  \langle \id_{\interp{[[D]]}}, \interp{[[t]]} \rangle;\mathsf{case}_{\interp{[[X]]}} : \interp{[[D]]} \to \interp{[[X]]}
  \]
  Thus, we obtain our desired equality because $\interp{[[D |-P 0 : Nat]]} = \t_{\interp{[[D]]}};\mathsf{z}$:
  \[
  \begin{array}{lll}
    \langle \id_{\interp{[[D]]}}, \t_{\interp{[[D]]}};\mathsf{z} \rangle;\case_{\interp{[[X]]}}
    & = & \langle \id_{\interp{[[D]]}}, \t_{\interp{[[D]]}} \rangle;(\id_{\interp{[[D]]}} \times \mathsf{z});\case_{\interp{[[X]]}}\\
    & = & \langle \id_{\interp{[[D]]}}, \t_{\interp{[[D]]}} \rangle;\pi_1;\interp{[[t1]]}\\
    & = & \interp{[[t1]]}.\\
  \end{array}
  \]
\item[] Case.\ \\ 
  \begin{center}
    \begin{math}
      $$\mprset{flushleft}
      \inferrule* [right=\ifrName{CaseS}] {
        [[D |-P n : Nat]]
        \\\\
            [[D |-P t1 : X && D, x : Nat |-P t2 : X]]
      }{[[D |-P case (suc n) {0 -> t1, suc x -> t2} = [n/x]t2 : X]]}
    \end{math}
  \end{center}
  We have the following morphisms by part one:
  \[
  \begin{array}{c}
    \interp{[[D]]} \mto^{\interp{[[n]]}} [[Nat]]\\
    \\
    \interp{[[D]]} \mto^{\interp{[[t1]]}} \interp{[[X]]}\\
    \\
    \interp{[[D]]} \times [[Nat]] \mto^{\interp{[[t2]]}} \interp{[[X]]}
  \end{array}
  \]
  Then by the definition of SNNO we know there exists an unique
  morphism
  \[ [[Y x Nat]] \mto^{\mathsf{case}_{\interp{[[X]]}}} \interp{[[X]]}\]
  such that the following is one equation that holds:
  \[
  \begin{array}{lll}
    (\id_{\interp{[[D]]}} \times \mathsf{suc});\mathsf{case}_{\interp{[[X]]}} = \interp{[[t2]]}
  \end{array}
  \]
  
  We also have the following interpretation for any $\interp{[[D |-P t : Nat]]}$:
  \[
  \interp{[[case t {0 -> t1, suc x -> t2}]]} =
  \langle \id_{\interp{[[D]]}}, \interp{[[t]]} \rangle;\mathsf{case}_{\interp{[[X]]}} : \interp{[[D]]} \to \interp{[[X]]}
  \]
  We obtain our result because $\interp{[[D |-P suc n : Nat]]} = \interp{[[n]]};\mathsf{suc}$:
  \[
  \begin{array}{lll}
    \langle \id_{\interp{[[D]]}}, \interp{[[n]]};\mathsf{suc} \rangle;\mathsf{case}_{\interp{[[X]]}}
    &  = & \langle \id_{\interp{[[D]]}}, \interp{[[n]]} \rangle;(\id_{\interp{[[D]]}} \times \mathsf{suc});\mathsf{case}_{\interp{[[X]]}}\\
    &  = & \langle \id_{\interp{[[D]]}}, \interp{[[n]]} \rangle;\interp{[[t2]]}.
  \end{array}
  \]
\end{itemize}
% subsection proof_of_soundness_(theorem~\ref{theorem:p-soundness}) (end)

\subsection{Proof of Monadic Strength (Lemma~\ref{lemma:monadic-strength})}
\label{subsec:proof_of_monadic_strength}
The strength map is defined as follows:
  \[
  \begin{array}{lll}
    \st{A,B} = (\eta_{A} \otimes \id_{TB});\m{A,B} : A \otimes TB \mto T(A \otimes B)\\    
  \end{array}
  \]
  where $\m{A,B} : TA \otimes TB \mto T(A \otimes B)$ is the natural
  transformation arising from the fact that $T$ is a monoidal
  endofunctor on $\cat{C}$.

  The first two diagrams are straightforward to prove using the
  monoidal structure.  We give explicit proofs of the final two
  diagrams.

  \begin{itemize}
  \item[] Case 1. Monadic Join and Strength:
    \[
    \bfig
        \square|mmaa|/`->``->/<1200,500>[
          A \otimes T^2 B``
          T(A \otimes TB)`
          T^2(A \otimes B);`
          \st{A,TB}``
          T\st{A,B}]
        
    \square(1200,0)|mmma|/``->`->/<1200,500>[`
      A \otimes TB`
      T^2(A \otimes B)`
      T(A \otimes B);``
      \st{A,B}`
      \mu_{A \otimes B}]

    \morphism(0,500)<2400,0>[
      A \otimes T^2 B`
      A \otimes TB;
      \id_{A} \otimes \mu_B]
    \efig
    \]          
    
    The previous diagram commutes because the following diagram commutes:
    \begin{center}
      \rotatebox{90}{$
        \bfig
        \square|mmmm|/`->`->`/<3499,500>[
          A \otimes T^2 B`
          A \otimes TB`
          TA \otimes T^2 B`
          TA \otimes TB;`
          \eta_A \otimes \id_{T^2B}`
          \eta_A \otimes \id_{TB}`]

        \square(0,-500)|amma|<1350,500>[
          TA \otimes T^2 B`
          T^2A \otimes T^2B`
          T(A \otimes TB)`
          T(TA \otimes TB);
          T\eta_A \otimes \id_{T^2 B}`
          \m{A,TB}`
          \m{TA,TB}`
          T(\eta_A \otimes \id_{TB})]

        \dtriangle(2500,-500)|mma|/`->`->/<1000,500>[
          TA \otimes TB`
          T^2(A \otimes B)`
          T(A \otimes B);`
          \m{A,B}`
          \mu_{A \otimes B}]

        \morphism(1350,-500)<1151,0>[
          T(TA \otimes TB)`
          T^2(A \otimes B);
          T\m{A,B}]

        \morphism(1350,0)<2149,0>[
          T^2A \otimes T^2B`
          TA \otimes TB;
          \mu_A \otimes \mu_B]

        \morphism(0,500)<3499,0>[
          A \otimes T^2 B`
          A \otimes TB;
          \id_A \otimes \mu_B]

        \place(675,-250)[1]
        \place(2350,-250)[2]
        \place(1749.5,250)[3]
        \efig
        $}
    \end{center}
    Diagram 1 commutes by naturality of $\m{}$, diagram 2 commutes
    because $T$ is a monoidal endofunctor, and diagram 3 commutes by
    using both the monoidal structure of the monad, and the fact that
    tensor is a functor.
    
  \item[] Case 2. Monadic Unit and Strength:
    \[
    \bfig
    \qtriangle|amm|<1000,500>[
      A \otimes B`
      A \otimes TB`
      T(A \otimes B);
      \id_A \otimes \eta_B`
      \eta_{A \times B}`
      \st{A,B}]    
    \efig
    \]
    The previous diagram commutes because the following diagram
    commutes:
    \[
    \bfig
    \btriangle|mma|/->`->`<-/<1500,500>[
      A \otimes B`
      T(A \otimes B)`
      TA \otimes TB;
      \eta_{A \otimes B}`
      \eta_A \otimes \eta_B`
      \m{A,B}]
    \qtriangle|amm|/ ->``->/<1500,500>[
      A \otimes B`
      A \otimes TB`
      TA \otimes TB;
      \id_A \otimes \eta_B``
      \eta_A \otimes \id_{TB}]
    \efig
    \]
    The lower triangle commutes because $\eta$ is a monoidal natural
    transformation, and the upper triangle commutes, because tensor is
    a functor.
  \end{itemize}

  
% subsection proof_of_monadic_strength_(lemma~\ref{lemma:monadic-strength (end)
% section proofs (end)
